\documentclass{article}
\usepackage[margin=0.6in]{geometry}
\usepackage[utf8]{inputenc}
\usepackage{physics}
\usepackage{graphicx}
\usepackage{siunitx}
\usepackage{amsmath}
\usepackage{amssymb}
\usepackage[dvipsnames]{xcolor}
\usepackage[numbers,sort&compress]{natbib}
\usepackage{bm}
\usepackage{url}
\usepackage{hyperref}
\usepackage{parskip}
\usepackage{lineno}
\usepackage{float}
\linenumbers

\setlength\parindent{0pt}
\renewcommand{\baselinestretch}{1.5}

\usepackage{authblk}

\title{Optimal time-dependent deployments of climate control technologies}
\author[1,2]{Henri F. Drake\textsuperscript{*}}
\author[1]{Ron Rivest}
\author[1]{Alan Edelman}
\author[1]{John Deutch}
\affil[1]{Massachusetts Institute of Technology, Cambridge, MA, USA}
\affil[2]{Woods Hole Oceanographic Institution, Woods Hole, MA, USA}

\date{}             %% if you don't need date to appear
\setcounter{Maxaffil}{0}
\renewcommand\Affilfont{\itshape\small}

\begin{document}
\maketitle

\section{Model formulation}

\subsection{CO$_{2}$ concentrations: baseline emissions, emissions reductions, and negative emissions}

CO$_{2}$ concentrations start at $c_{0}(t_{0}) = \SI{415}{ppm}$ in $t_{0}=2020$ and increase according to the baseline emissions scenario $q(t)$ (in ppm / year), emissions reductions, and negative emissions. The baseline concentrations are simply $c_{0}(t) = c_{0}(t_{0}) + \int_{t'=t_{0}}^{t} q(t') \text{ d}t'$. Emissions reductions are parameterized by the fractional reduction of yearly emissions $\phi(t) \in [0,1]$ such that the annual emissions become $q(1-\phi(t))$. Negative emissions are parameterized by the fraction $\varphi(t) \in [0,1]$ of the baseline emissions $q(t_{0}) = \SI{5}{ppm/year}$ in 2020 that are sequestered from the atmosphere in a given year (either by direct air capture, afforestation, or otherwise). The CO$_{2}$ concentrations in a given year are thus given by:
\begin{equation}
    c_{\phi, \varphi}(t) = c_{0}(t_{0}) + \int_{t'=t_{0}}^{t} q(t')(1-\varphi(t')) - q(t_{0}) \phi(t') \text{ d}t'.\label{eq-CO2-conc}
\end{equation}

\subsection{Temperature response}
The linearized energy balance model underlying the model for the temperature response is:
\begin{equation}
    c_{f} \dv{T}{t} = -BT + F(t),
\end{equation}
where $c_{f}$ is the thermal heat capacity of the climate system, $B$ is the sensitivity parameter, $T$ is the temperature anomaly relative to a preindustrial mean (1850 to 1900), and $F(t)$ is an anthropogenic CO$_{2}$ forcing. We assume the thermal heat capacity of the climate system is sufficiently small such that $T(t)=F(t)/B$ at all times. We determine the sensitivity parameter $B$ by tuning it to the best-guess estimate of the equilibrium climate sensitivity $T_{\text{ECS}} = F_{2\times}/B = \SI{3}{\celsius}$. Since $B = F_{2\times}/T_{\text{ECS}} = F(t)/T(t)$, we have $T(t) = T_{\text{ECS}} \frac{F(t)}{F_{2\times}}$. The greenhouse gas forcing due to CO$_{2}$ is well known to be a logarithmic function of concentration, $F(t)-F(t_{0}) = b \log(\frac{c(t)}{c(t_{0})})$.

The baseline temperature response to the baseline CO$_{2}$ concentrations is 
\begin{equation}
    T_{0}(t) = 
    T_{0}(t_{0}) + 
    \frac{T_{\text{ECS}}}{\log(2)} \log(\frac{c_{0}(t)}{c_{0}(t_{0})}) = 
    T_{0}(t_{0}) + 
    \epsilon \log(\frac{c_{0}(t)}{c_{0}(t_{0})}),
\end{equation}
where $\epsilon \equiv T_{\text{ECS}} / \log(2)$ is the transient warming parameter, while for the CO$_{2}$ concentration scenario $c_{\phi,\varphi}(t)$ it is given by
\begin{equation}
    T_{\phi,\varphi}(t) =
    T_{0}(t_{0}) + 
    \frac{T_{\text{ECS}}}{\log(2)} \log(\frac{c_{\phi,\varphi}(t)}{c_{0}(t_{0})}) = 
    T_{0}(t_{0}) + 
    \epsilon \log(\frac{c_{\phi,\varphi}(t)}{c_{0}(t_{0})}).
\end{equation}


The temperature response can be further modified by solar radiation management or solar geongineering, which is parameterized by reducing the emissions-controlled temperature response $T_{\phi,\varphi}$ by a fraction $\lambda(t) \in [0,1]$, such that the controlled temperature response is given by:
\begin{equation}
    T_{\phi, \varphi, \lambda} = \left(
    T_{0}(t_{0}) + 
    \epsilon \log(\frac{c_{\phi,\varphi}(t)}{c_{0}(t_{0})}) 
    \right) (1 - \lambda(t)).
\end{equation}

\subsection{Climate damages}

Yearly climate damages are assumed to be of the quadratic form $D(t) = \beta T(t)^{2}$, such that successive temperature increases are increasingly damaging. The baseline climate damages are thus
\begin{equation}
    D_{0}(t) = \beta T_{0}(t)^{2}.
\end{equation}
Adaptation to climate change impacts (e.g. building sea walls, installing air conditioning units, planting climate-resilient crops) is parameterized by reducing yearly damages by a fraction $\chi(t) \in [0,1]$, such that the total controlled damages are:
\begin{equation}
    D_{\phi, \varphi, \lambda, \chi} = \beta \; (T_{\phi, \varphi, \lambda}(t))^{2} \; (1-\chi(t)).
\end{equation}

\subsection{Control costs}

The yearly cost of climate control technologies are given by
\begin{equation}
    C(t) = \sum_{\alpha \in \mathcal{A}} C_{\alpha} f(\alpha(t)),
\end{equation}
where $\mathcal{A} = \{ \phi, \varphi, \lambda, \chi \}$ is the set of climate control technologies considered here, $C_{\alpha}$ is reference cost of each climate control technology, and $f(\alpha)$ is a function that determines how the deployment cost increases as a function of fractional deployment. Here, we will focus on the medium deployment cost scenario $f_{\text{med}}(\alpha) = \alpha^{2}$, which has the following convenient properties: 
\begin{itemize}
    \item $\left. \dv{f}{\alpha}\right|_{\alpha=0} = 0$ (initial deployment is effectively free),
    \item $f(1) = 1$ (full deployment costs $C_{\alpha}$), and
    \item $\dv[2]{f}{\alpha} > 0$ (deployment gets progressively more and more expensive).
\end{itemize}
We also leave open the possibility of low cost $f_{\text{low}} = \left( \frac{\alpha}{1+\alpha} \right)^{2}$ and high cost $f_{\text{high}} = \left( \frac{\alpha}{1-\alpha} \right)^{2}$ deployment scenarios but do not explore them further here. 

\subsection{Total costs and discounting}

The total yearly costs of climate change are the sum of the (uncontrolled or controlled) climate damages and the costs of control technologies. A common economic assumption is that society discounts future costs relative to present costs by a multiplicative factor $(1 + \rho)^{-(t-t_{0})}$, determined by the utility discount rate $\rho$. The discounted total cost (in dollars) of climate change from $t_{0}$ to $t_{f}$ is thus given by:
\begin{equation}
    \mathcal{T} =
    \int_{t'=t_{0}}^{t_{f}}
    (C(t') + D_{\mathcal{A}}(t')) (1 + \rho)^{-(t'-t_{0})} \text{ d}t'
\end{equation}

In contrast to typical Integrated Assessment Models, which follow classic economic theories of optimal economic growth and solve for the maximal welfare based on the discounted utility of consumption, we here completely ignore economic feedbacks and simply aim to minimize the total discounted cost of climate change. Here, the total discounted cost of climate change include both the direct cost of damages from climate impacts (e.g. increased mortality from heat waves or property damage from higher storm surges) and the indirect cost of deploying climate controls which aim to alleviate the direct costs of climate impacts. The goal is to find the trajectories of control deployments which optimize the trade-off between the costs of climate impacts and control deployments.

One argument in favor of this simplified approach to the economics is the apparent insensitivity of economic output and consumption to climate policies in IAMs such as the DICE2013 model \cite[][figures 2 and 3]{Nordhaus}.

\subsection{Optimization Method}

As a crude first attempt at optimizing the control parameters $\mathcal{A} = \{\phi, \varphi, \lambda, \chi\}$, we use a simple gradient descent algorithm (with momentum) to find the minimum of the modified cost function

\begin{equation}
    \mathcal{T}_{M} =
    \int_{t'=t_{0}}^{t_{f}}
    (C(t') + D_{\mathcal{A}}(t')) (1 + \rho)^{-(t'-t_{0})} \text{ d}t'
    + \tau \sum_{\alpha \in \mathcal{A}} \left( \sum_{t} \left(\pdv{\alpha}{t} \right)^{2} + \alpha(t_{0})^{2} \right),
\end{equation}
where $\tau = 200$ is large relative to $\mathcal{T}$ and the new terms are ad-hoc modifications that enforce additional constraints on the optimization: the first term in the sum effectively enforces continuity of the control parameters in time and the second term asserts that there is no deployment of climate control technologies , $\alpha(t_{0})=0$.

\section{Next steps}

\subsection{Improved optimization algorithm}

I am new to optimization and did the simplest thing I could think of. The primary downsides of this approach are: 1) it assumes we reach the first minimum we reach from our initial guess is the global minimum (perhaps this is a fine assumption for this relatively small number of dimensions and smooth cost function); 2) it is slow (~20 seconds per simulation), making large ensemble simulations (e.g. stochastic or monte carlo) intractable for the time being; 3) the imposed constraints are ad-hoc and somewhat arbitrary (is there a better way of imposing continutiy and initial / boundary conditions?).

\subsection{Imposing a budget constraint}

I have taken the approach of minimizing the total cost of climate change, including both the cost due to (controlled or uncontrolled) climate damages and the cost of deploying climate control technologies. Jon Deutch's approach was somewhat different: he prescribed a total budget and aimed to minimize the controlled climate damages within the prescribed budget. I think both are equally interesting.

\subsection{Transient heat and carbon uptake}
Anthropogenically-emitted carbon and heat trapped by anthropogenically-emitted carbon do not stay in the atmosphere once emitted. Over the last hundred of so years, $90\%$ of anthropogenic heat and $30\%$ of anthrogenic carbon have been absorbed by the ocean, by exchange with atmosphere at the sea surface and subsequent turbulent transfer into the deep ocean. \textcolor{BlueViolet}{[About $20\%$ of the carbon was also absorbed by the biosphere, but for our purposes we can probably ignore this process).]}

We are effectively ignoring both of these important transient uptake processes and could incorporate both with simple two-box models.

\subsection{Monte Carlo simulations}

Our model currently doesn't take into account any uncertainty or stochastic effects. One could imagine myriad additional complexities, but some a physical climate scientists' perspective, one of the most interesting additions would be uncertainty in the equilibrium climate sensitivity. The consensus view is that the equilibrium climate sensitivity is between about $\SI{1.5}{\celsius}$ and $\SI{4.5}{\celsius}$. Given the non-linearities in the climate damage and deployment cost functions, the projection of uncertainties in the equilibrium climate sensitivity onto the optimal controls may be interesting.

%\bibliographystyle{naturemag}
%\bibliography{climate-model-performance.bib,manual_refs.bib}

\end{document}